% Anthony Gargiulo's Resume
% Based on the template by 
% Andrew McNabb
% Created: Fri Aug 12 11:11:56 EDT 2011
% Last Modified: Fri Aug 12 11:12:06 EDT 2011

\documentclass[11pt,oneside]{article}
\usepackage{geometry}
\usepackage[T1]{fontenc}

\pagestyle{empty}
\geometry{letterpaper,tmargin=1in,bmargin=1in,lmargin=1in,rmargin=1in,headheight=0in,headsep=0in,footskip=.3in}

\setlength{\parindent}{0in}
\setlength{\parskip}{0in}
\setlength{\itemsep}{0in}
\setlength{\topsep}{0in}
\setlength{\tabcolsep}{0in}

% Name and contact information
\newcommand{\name}{Anthony Gargiulo}
\newcommand{\addr}{20 Kunen Ave, Bethpage, NY 11714}
\newcommand{\phone}{(515) 236-5102}
\newcommand{\email}{agargiulo@anthonygargiulo.info}


%%%%%%%%%%%%%%%%%%%%%%%%%%%%%%%%%%%%%%%%%%%%%%%%%%%%%%%%%
% New commands and environments

% This defines how the name looks
\newcommand{\bigname}[1]{
	\begin{center}\fontfamily{phv}\selectfont\Huge\scshape#1\end{center}
}

% A ressection is a main section (<H1>Section</H1>)
\newenvironment{ressection}[1]{
	\vspace{4pt}
	{\fontfamily{phv}\selectfont\Large#1}
	\begin{itemize}
	\vspace{3pt}
}{
	\end{itemize}
}

% A resitem is a simple list element in a ressection (first level)
\newcommand{\resitem}[1]{
	\vspace{-4pt}
	\item \begin{flushleft} #1 \end{flushleft}
}

% A ressubitem is a simple list element in anything but a ressection (second level)
\newcommand{\ressubitem}[1]{
	\vspace{-1pt}
	\item \begin{flushleft} #1 \end{flushleft}
}

% A resbigitem is a complex list element for stuff like jobs and education:
%  Arg 1: Name of company or university
%  Arg 2: Location
%  Arg 3: Title and/or date range
\newcommand{\resbigitem}[3]{
	\vspace{-5pt}
	\item
	\textbf{#1}---#2 \\
	\textit{#3}
}

% This is a list that comes with a resbigitem
\newenvironment{ressubsec}[3]{
	\resbigitem{#1}{#2}{#3}
	\vspace{-2pt}
	\begin{itemize}
}{
	\end{itemize}
}

% This is a simple sublist
\newenvironment{reslist}[1]{
	\resitem{\textbf{#1}}
	\vspace{-5pt}
	\begin{itemize}
}{
	\end{itemize}
}



%%%%%%%%%%%%%%%%%%%%%%%%%%%%%%%%%%%%%%%%%%%%%%%%%%%%%%%%%
% Now for the actual document:

\begin{document}

\fontfamily{ppl} \selectfont

% Name with horizontal rule
\bigname{\name}

\vspace{-8pt} \rule{\textwidth}{1pt}

\vspace{-1pt} {\small\itshape \addr \hfill \phone; \email}

\vspace{8 pt}




%%%%%%%%%%%%%%%%%%%%%%%%
\begin{ressection}{Education}

	\begin{ressubsec}{Rochester Institute of Technology}{Rochester, NY}{Majoring in Computer Science}
		\ressubitem{Projected Graduation Date: May, 2015}
	\end{ressubsec}

\end{ressection}


%%%%%%%%%%%%%%%%%%%%%%%%
\begin{ressection}{Experience}

	\begin{ressubsec}{IKEA Long Island}{Hicksville, NY}{Cashier: December 2009--August 2010, June 2011--August 2019}
		\ressubitem{Worked as a cashier, and other positions when that department was short on people.}
	\end{ressubsec}

	\begin{ressubsec}{Bethpage School District}{Bethpage, NY}{Teacher's Assistant: July 2009}
		\ressubitem{Assited teacher with normal classroom duties and helped with the lessons.}
		\ressubitem{Class was a video game design style class, aimed at middle school ages, and involved the use of MIT Scratch and Game Maker.}
	\end{ressubsec}

	\begin{ressubsec}{Bethpage School District}{Bethpage, NY}{Summer Camp Teacher: 2011}
		\ressubitem{Taught 4th to 7th grade students in a "Video Game Design" class for a summer tech camp.}
		\ressubitem{Involved making lesson plans, and teaching topics ranging from simple flow control to variable use}
		\ressubitem{Class was taught using a spin of MIT Scratch named BYOB (Build Your Own Blocks)}
	\end{ressubsec}

\end{ressection}


%%%%%%%%%%%%%%%%%%%%%%%%
\begin{ressection}{Skills}

	\resitem{\textbf{Operating Systems:} Linux (Fedora, Debian, some Gentoo), Windows XP/7}

	\begin{reslist}{Computer Languages:}

		\ressubitem{Proficient in Java, Python, Perl}

		\ressubitem{Familiar with HTML, \LaTeX, SQL, PHP, bash, zsh, and other UNIX shells}

	\end{reslist}

	\begin{reslist}{Tools and Systems:}

		\ressubitem{Proficient in Apache, vim, Subversion, Git}

		\ressubitem{Familiar with Cisco IOS, MySQL, PostgreSQL, OpenLDAP, Samba}

	\end{reslist}


\end{ressection}


%%%%%%%%%%%%%%%%%%%%%%%%


\end{document}
